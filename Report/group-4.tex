\documentclass[a4paper]{article}

%% Language and font encodings
\usepackage[english]{babel}
\usepackage[utf8x]{inputenc}
\usepackage[T1]{fontenc}
\usepackage{fancyhdr}
\usepackage[at]{easylist}
\usepackage{mathptmx}
\usepackage{bm}
\usepackage{float}
\usepackage{subcaption}
\usepackage{listings}
\usepackage{courier}
\usepackage{siunitx}

%% Sets page size and margins
\usepackage[a4paper,top=3cm,bottom=2cm,left=3cm,right=3cm,marginparwidth=1.75cm]{geometry}
\setlength{\parindent}{0pt}

%% Set sections to 1.a etc
\renewcommand{\thesubsection}{\thesection.\alph{subsection}}

%% Useful packages
\usepackage{amsmath}
\usepackage{amssymb}
\usepackage{graphicx}
\usepackage[colorinlistoftodos]{todonotes}
\usepackage[colorlinks=true, allcolors=blue]{hyperref}
\usepackage{algpseudocode}

\title{ENGSCI 355 Project 1}
\author{Andrew Jackson\\Logan Wu\\Scott Sung}
\makeatletter

\setlength{\parskip}{\baselineskip}
\fancyhead{}
\fancyhead[L]{\@title}
\fancyhead[R]{Jackson, Sung, Wu}
\newlength{\drop}
\begin{document}
\begin{titlepage}
    \drop=0.1\textheight
    \centering
    \vspace*{\baselineskip}
    \rule{\textwidth}{1.6pt}\vspace*{-\baselineskip}\vspace*{2pt}
    \rule{\textwidth}{0.4pt}\\[\baselineskip]
    {\LARGE \@title}\\[0.2\baselineskip]
    \rule{\textwidth}{0.4pt}\vspace*{-\baselineskip}\vspace{3.2pt}
    \rule{\textwidth}{1.6pt}\\[\baselineskip]
    \vspace*{2\baselineskip}
    {\Large \textsc{\@author}\par}
    {\itshape Department of Engineering Science\par}
    \vspace*{2\baselineskip}
    {\scshape \today} \        {\large The University of Auckland}\par
\end{titlepage}
\makeatother
\pagestyle{fancy}

\section{Formulation}

Our model can be conceptualised as two, linked linear programs. The first is a roster of shifts, and the second one is a cyclical network representing ward occupancies during the 42 days of the roster.

\subsection{Parameters}

\begin{align*}
S &= \{\text{A, P, N, Z, X, O}\}; \quad\text{Set of shift types}\\
W &= \{1, \dots, 6\}; \quad\text{Set of weeks in the roster cycle}\\
D &= \{\text{Mon},\dots,\text{Sun}\}; \quad\text{Set of days in a week}
\end{align*}

Note that for continuity, a dummy week $0$ and dummy day $\text{Sun}_\text{dummy}$ also exist but are not part of the sets $W$ and $D$.

\subsection{Decision Variables}

$X$ is the array of binary variables determining if a type of shift belongs to a given week and day in the roster:
$$x_{s, w, d} \in \{0, 1\} \quad\forall s\in S,\  w\in W \cup \{0\},\  d\in D \cup \{\text{Sun}_\text{dummy}\}$$

$Y$ determines if a given week in the roster is the night shift week. Note the night shift also includes the final three days of the preceding week:
$$y_w \in \{0, 1\} \quad\forall w\in W \cup \{\text{Sun}_\text{dummy}\}$$

$V$ denotes whether registrars are forced to take a weekend off:
$$v_w \in \{0, 1\} \quad\forall w\in W \cup \{\text{Sun}_\text{dummy}\}$$

\subsection{Constraints}

Create a dummy week 0 that is equal to the final week.
\begin{align}
  x_{s, 0, d} = x_{s, |W|, d} \quad\forall s\in S,\ d\in D
\end{align}

Create a dummy day 0 that is equal to Sunday of the previous week, allowing wrap-around from Sunday to Monday.
\begin{align}
  x_{s, w-1, |D|} = x_{s, w, 0} \quad\forall s\in S,\ w\in W
\end{align}

Ensure every slot has a shift assigned by summing over all shift types, except for the night-shift week which must have two.
\begin{align}
  \sum_{s\in S} x_{s, w, d} - y_w = 1 \quad\forall w\in W,\ d\in D
\end{align}

Every day must have a single registrar assigned to each A, P and N shift.
\begin{align}
  \sum_{w\in W} x_{s, w, d} = 1 \quad\forall s\in \{\text{A, P, N}\},\ d\in D
\end{align}

Every P shift must follow an A shift, except for Sunday where an A must follow.
\begin{align}
  x_{\text{P}, w, d} &= x_{\text{A}, w, d-1} \quad\forall w\in W,\ d\in D\\
  x_{\text{A}, w, d} &= x_{\text{A}, w, d-1} \quad\forall w\in W,\ d\in \{\text{Sun}\}
\end{align}

Setting up the night shift: only one week can be the full `night shift' week.
\begin{align}
  \sum_{w\in W} y_w = 1
\end{align}

The Friday to Sunday before the full night shift week are also night shifts.
\begin{align}
  x_{\text{N}, w-1, d} - y_w = 0 \quad\forall w\in W,\ d\in \{\text{Fri, Sat, Sun}\}
\end{align}

The Monday to Thursday of the full night shift week are night shifts.
\begin{align}
  x_{\text{N}, w, d} - y_w = 0 \quad\forall w\in W,\ d\in \{\text{Mon, Tue, Wed, Thu}\}
\end{align}

The Friday to Sunday of the full night shift week are sleep shifts.
\begin{align}
  x_{\text{Z}, w, d} - y_w = 0 \quad\forall w\in W,\ d\in \{\text{Fri, Sat, Sun}\}
\end{align}

Ensure no one else has a sleep shift and is slacking off!
\begin{align}
  \sum_{w\in W,\ d\in D} x_{\text{Z}, w, d} = \text{n.o. allowed rests (i.e., 3)}
\end{align}

To give people weekends off: no weekdays are allowed to be taken off:
\begin{align}
  x_{\text{X}, w, d} = 0 \quad\forall w\in W,\ d\in \{\text{Mon, \dots, Fri}\}
\end{align}

Weekends must be taken off if scheduled as a 'weekend off' (but may be taken off on other weekends):
\begin{align}
  x_{\text{X}, w, d} \ge 0 \quad\forall w\in W,\ d\in \{\text{Sat, Sun}\}
\end{align}

No two consecutive weekends can pass without a weekend off being forced.
\begin{align}
  v_w + v_{w-1} \ge 1 \quad\forall w\in W
\end{align}

\end{document}

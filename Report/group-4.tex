\documentclass[a4paper]{article}

%% Language and font encodings
\usepackage[english]{babel}
\usepackage[utf8x]{inputenc}
\usepackage[T1]{fontenc}
\usepackage{fancyhdr}
\usepackage[at]{easylist}
\usepackage{mathptmx}
\usepackage{bm}
\usepackage{float}
\usepackage{subcaption}
\usepackage{listings}
\usepackage{courier}
\usepackage{siunitx}
\usepackage{framed}
\usepackage{multicol}
\usepackage[shortlabels]{enumitem}

\usepackage[bottom]{footmisc}

%% Sets page size and margins
\usepackage[a4paper,top=3cm,bottom=2cm,left=3cm,right=3cm,marginparwidth=1.75cm]{geometry}
\setlength{\parindent}{0pt}

%% Set sections to 1.a etc
\renewcommand{\thesubsection}{\thesection.\alph{subsection}}

%% Useful packages
\usepackage{amsmath}
\usepackage{amssymb}
\usepackage{graphicx}
\usepackage[colorinlistoftodos]{todonotes}
\usepackage[colorlinks=false, allcolors=black]{hyperref}
\usepackage{algpseudocode}
\usepackage{amsmath}
\usepackage{amsfonts}
\usepackage{amssymb}
\usepackage{mathtools}
\usepackage{xcolor}
\definecolor{NAVY}{rgb}{0.2,0.2,1}
\definecolor{YELLOW}{rgb}{1,0.8,0}
\definecolor{LIME}{rgb}{0.5,0.8,0}
\DeclarePairedDelimiter{\ceil}{\lceil}{\rceil}

\title{On Improving Hospital Service for Americans and Russians}
\author{Logan Wu\\Andrew Jackson\\Scott Sung}
\makeatletter

\setlength{\parskip}{\baselineskip}
\fancyhead{}
\fancyhead[L]{\@title}
\fancyhead[R]{Wu, Jackson, Sung}
\newlength{\drop}
\begin{document}
\begin{titlepage}
    \drop=0.1\textheight
    \centering
    \vspace*{\baselineskip}
    \rule{\textwidth}{1.6pt}\vspace*{-\baselineskip}\vspace*{2pt}
    \rule{\textwidth}{0.4pt}\\[\baselineskip]
    {\LARGE \@title}\\[0.2\baselineskip]
    \rule{\textwidth}{0.4pt}\vspace*{-\baselineskip}\vspace{3.2pt}
    \rule{\textwidth}{1.6pt}\\[\baselineskip]
    \vspace*{2\baselineskip}
    {\Large \textsc{\@author}\par}
    {\itshape Department of Engineering Science\par}
    \vspace*{2\baselineskip}
    {\scshape \today} \        {\large The University of Auckland}\par
    \vspace{\fill}
    \begin{abstract}
    In this report, we describe the mixed-integer formulation to balance patient numbers at a hospital by rostering the registrars in each ward. We predict the infinite-horizon mean occupancies of each ward assuming a Poisson distribution of discharges.
    
We estimate that our program can reduce the average daily imbalance (the difference in patients between the most and least occupied wards each day) by approximately 23\% compared to a random, unoptimised roster. A pairwise analysis of the results reveals that even in the scenario where registrars cannot change their starting weeks, the optimisation program is still able to significantly reduce imbalances.
    \end{abstract}
    \vspace{\fill}
\end{titlepage}
\makeatother
\pagestyle{fancy}

\section{Introduction}

This linear program generates feasible rostering solutions with the aim of minimising the expected disparity between the occupancies of the hospital's wards. The program is flexible, easily accommodating different staff arrangements and historical data with small modifications to a data file.

\section{Assumptions}
\begin{multicols}{2}
\subsection{Independent, Identically Distributed People}

To calculate the outcome of a roster, we take advantage of symmetry in the problem and take the liberty of adding some when convenient.

We assume that all ward teams act equally and independently, i.e. there is no difference if we swap the Navy and Lime teams. This is a fair assumption because registrars usually all have the same background and ward teams are physically separated.

A trickier assumption is that all patients are independent and memoryless. This allows us to model them with a Poisson distribution where the mean stay is 2.1 days (Simulation Project, ENGSCI 355).
\iffalse\footnote{Weiss, A. J. and Elixhauser, A. (2014). \emph{Overview of Hospital Stays in the United States, 2012}.\\Retrieved from \emph{https://www.hcup-us.ahrq.gov/reports/statbriefs/sb180-Hospitalizations-United-States-2012.pdf}}\fi In reality this should be a heavy-tail distribution (if a patient is going to stay for a while, they will be there for a \emph{long} time), but the memoryless property is a convenient approximation. It also makes the hospital stable, so over a long period of time the number of discharges converges to the number of admissions (everyone is eventually accounted for).

\iffalse
\subsection{Occupation Numbers are Periodic}

We assume that the mean admission for each day of the week does not change between weeks, so our admissions data is an average of each day from the past 15 weeks.

We also assume that patient numbers are periodic over the six-week roster. This avoids any discontinuities and checks that the hospital is stable, although this is guaranteed anyway by the Poisson discharge model.
\fi
\subsection{Random Admissions \& Lengths of Stays}

We assume that there is no trend in the mean admissions between weeks -- even though there will be random fluctuations, numbers will be centered on a mean. This allows us to model admissions as an average of each weekday for the past 15 weeks.

If there is no trend in the length of stay, we can model discharges with a probability distribution. This might be violated, for example, if doctors get faster at treating patients with more experience.

\end{multicols}
\section{Objective}

We have defined our objective as minimising the average expected disparity between the most and least occupied wards. We average over a looping, 42-day cycle. This estimate gives us a statistical average for the overall perceived difference in ward occupant numbers. We choose to minimise the average over the whole roster rather than the peaks for two reasons:
\begin{enumerate}[noitemsep, topsep=-6pt]
\item This objective drives the program to minimise the difference over every single day, rather than focusing only on peak differences; the hospital should notice a constant improvement.
\item Peaks are far more difficult to control and cannot be reliably reduced ahead of time due to their random nature.
\end{enumerate}

\section{Results}
\begin{table}[h]
\centering
\caption{Final optimised roster. The colour codes are the initial starting weeks for each registrar. Notably, each ward has its registrars three weeks apart.}
\label{tab:roster}
\begin{tabular}{r|ccccccc}
Week	&Mon	&Tue	&Wed	&Thu	&Fri	&Sat	&Sun	\\
\hline
{\color{LIME}$\bullet_1$} 1	& O	& O	& O	& O	& O	& A	& A	\\
{\color{NAVY}$\bullet_1$} 2	& P	& O	& O	& A	& P	& X	& X	\\
{\color{YELLOW}$\bullet_1$} 3	& O	& O	& O	& O	& A	& P	& X	\\
{\color{LIME}$\bullet_2$} 4	& A	& P	& A	& P	& O	& X	& X	\\
{\color{NAVY}$\bullet_2$} 5	& O	& O	& O	& O	& N	& N	& N	\\
{\color{YELLOW}$\bullet_2$} 6	& NO	& NA	& NP	& NO	& ZO	& ZX	& ZX	\\
\end{tabular}
\end{table}

By comparing a series of random, unoptimised rosters ($N=30$, obtained by disabling the objective function) with our roster (Table \ref{tab:roster}), we observe a significant improvement in the daily ward imbalances (Figure \ref{fig:comparison}). On a day-to-day basis using a random roster, the most occupied ward will have an average of between 34 and 36 people ($\alpha = 0.05$) more people than the least occupied. After optimisation, this decreased to 27 -- an estimated reduction of 8 people or 23\%.

While investigating the relative effects, we found no evidence that the 1\&4, 2\&5, 3\&6 week arrangement was significant. However, a pairwise analysis before and after adding the objective function (each pair with the same starting week arrangement) found a difference of between 5 and 7 fewer people per day. This shows that even if the starting weeks are fixed, optimisation still provides a significant improvement.

\begin{figure}[h]
    \centering\includegraphics[width=\linewidth]{../results/comparison}
    \caption{Effect of optimisation on day-to-day patient imbalances}
    \label{fig:comparison}
\end{figure}

\section{Conclusion}

Our program significantly reduces hospital ward imbalances compared to a random roster by between 7 and 9 people on average. This results in improved in patient care as registrars are more evenly distributed between patients.

Because we are using predictions for the expected patient numbers, our method results in a constant, average improvement every day; not only will the hospital notice a general improvement, peak discrepancies will also be lower than before.

\newpage
\appendix
\iffalse
\section{Calculating Expected Number of Patients}

We have been supplied with 15 weeks of exact data detailing admissions and discharges in fine granularity. We use this data to calculate expectation values of admissions, but ignore discharge data in lieu of our own exponential patient discharge model. By doing this, we can generate unbiased statistical estimates for mean admissions on each weekday, which is necessary to calculate average patient numbers.

We first need to look at some expectation identities that have been used implicitly in the formulation. Let X, Y and Z be random variables, such as the population in a ward at any given time or the admissions on any given day:
\begin{enumerate}
\item $E[X] \equiv E[\bar{X}]$; the expectation of the random variable is equal to the expectation of its mean. In our model, we sum our expectation values linearly; if $Z= X + Y$,\quad $E[Z] = E[X] + E[Y]$.
\item We use the fact that we are modelling a time-discounted, infinite horizon stochastic network where $x$ is the probability of a given patient leaving and $x\sim \lambda e^{-\lambda x}$. The ``sample size" -- the number of 42-day rosters to average over -- is $\infty$ when applied to an infinite cycle. Therefore, by the central limit theorem, $E[\bar{x}] \equiv 1/\lambda \implies E[A] \equiv 1/\lambda \equiv \rho$.
\item Finally, applied to a stochastic, exponentially decaying infinite horizon network, we can say about the expected number of occupants on day $i$: $E[x_i] \equiv (1-\rho) \times E[x_{i-1}] + E[\text{admissions}_{i-1}]$, and this is the formula that appears in Equation 15.
\end{enumerate}

By approximating patient lengths of stays as exponential and performing calculations in expectation values, we can quickly and efficiently (6 seconds total for 15 enumerations) predict the mean number of patients at any point within the 42-day roster.
\fi
\section{Formulation}

Our model can be conceptualised as two coupled linear programs. The first is a roster of shifts to ensure feasibility, and the second one is a cyclical network representing ward occupancies during the 42 days of the roster.

\subsection{Parameters}
\begin{multicols}{2}
\noindent


$\mathbb{S} = \{\text{A, P, N, Z, X, O}\}$ \dotfill Set of shift types\\
$\mathbb{W} = \{1, \dots, 6\}$ \dotfill Set of weeks in the roster cycle\\
$\mathbb{D} = \{\text{Mon},\dots,\text{Sun}\}$ \dotfill Set of days in a week\\
$\mathbb{Y}  = \{0, 0,\dots, 0, 1\}$ \dotfill Night shift on the last week

For continuity, a dummy week $0$ and dummy day $\text{Sun}_\text{dummy}$ before Monday also exist but are not part of the sets $\mathbb{W}$ and $\mathbb{D}$. Also, fixing the night shift week takes advantage of symmetry between the weeks.
\end{multicols}
\subsection{Decision Variables}

$X$ is the array of binary variables determining if a type of shift belongs to a given week and day in the roster:
$$x_{s, w, d} \in \{0, 1\} \quad\forall s\in \mathbb{S},\  w\in \mathbb{W} \cup \{0\},\ d\in \mathbb{D} \cup \{\text{Sun}_\text{dummy}\}$$
$V$ denotes whether registrars are forced to take a weekend off:
$$v_w \in \{0, 1\} \quad\forall w\in \mathbb{W} \cup \{\text{Sun}_\text{dummy}\}$$

\subsection{Constraints}

Create a dummy week 0 that is equal to the final week.
\begin{align}
  x_{s, 0, d} = x_{s, |\mathbb{W}|, d} \quad\forall s\in \mathbb{S},\ d\in \mathbb{D}
\end{align}
Create a dummy day 0 that is equal to the last day of the previous week, allowing wrap-around from Sunday to Monday.
\begin{align}
  x_{s, w-1, |\mathbb{D}|} = x_{s, w, 0} \quad\forall s\in \mathbb{S},\ w\in \mathbb{W}
\end{align}
Ensure every slot has a shift assigned by summing over all shift types, except for the night-shift week which must have two.
\begin{align}
  \sum_{s\in \mathbb{S}} x_{s, w, d} - y_w = 1 \quad\forall w\in \mathbb{W},\ d\in \mathbb{D}
\end{align}
Every day must have a single registrar assigned to each A, P and N shift.
\begin{align}
  \sum_{w\in \mathbb{W}} x_{s, w, d} = 1 \quad\forall s\in \{\text{A, P, N}\},\ d\in \mathbb{D}
\end{align}
Every P shift must follow an A shift, except for Sunday where an A must follow an A.
\begin{align}
  x_{\text{P}, w, d} &= x_{\text{A}, w, d-1} \quad\forall w\in \mathbb{W},\ d\in \mathbb{D}\\
  x_{\text{A}, w, d} &= x_{\text{A}, w, d-1} \quad\forall w\in \mathbb{W},\ d\in \{\text{Sun}\}
\end{align}
The Friday to Sunday before the full night shift week are also night shifts.
\begin{align}
  x_{\text{N}, w-1, d} - y_w = 0 \quad\forall w\in \mathbb{W},\ d\in \{\text{Fri, Sat, Sun}\}
\end{align}
The Monday to Thursday of the full night shift week are night shifts.
\begin{align}
  x_{\text{N}, w, d} - y_w = 0 \quad\forall w\in \mathbb{W},\ d\in \{\text{Mon, Tue, Wed, Thu}\}
\end{align}
The Friday to Sunday of the full night shift week are sleep shifts.
\begin{align}
  x_{\text{Z}, w, d} - y_w = 0 \quad\forall w\in \mathbb{W},\ d\in \{\text{Fri, Sat, Sun}\}
\end{align}
Ensure no one else has a sleep shift and is slacking off!
\begin{align}
  \sum_{w\in W,\ d\in \mathbb{D}} x_{\text{Z}, w, d} = 3 \text{ (number of allowed rests)}
\end{align}
No weekdays are allowed to be taken off.
\begin{align}
  x_{\text{X}, w, d} = 0 \quad\forall w\in \mathbb{W},\ d\in \{\text{Mon, \dots, Fri}\}
\end{align}
Weekends must be taken off if scheduled as a 'weekend off' (but may be taken off on other weekends).
\begin{align}
  x_{\text{X}, w, d} \ge 0 \quad\forall w\in \mathbb{W},\ d\in \{\text{Sat, Sun}\}
\end{align}
No two consecutive weekends can pass without a weekend off being forced.
\begin{align}
  v_w + v_{w-1} \ge 1 \quad\forall w\in \mathbb{W}
\end{align}

\subsection{Objective Parameters}

There are several new definitions that are particular to this section of the formulation. This section can be represented as a 42-node cycle with discounted arcs between consecutive nodes, and artificial insertions on admission days averaged from the 15-week data.

\begin{multicols}{2}
\begin{flushright}
$\mathbb{WA} = \{\text{lime},\ \text{navy},\ \text{yellow}\}$ \dotfill Set of wards\\
$\mathbb{R} = \mathbb{W}\times \mathbb{D}$ \dotfill Set of all days in the roster, to simplify $r+1$ and $r-1$ subscripts\\
$\rho = 0.45$ \dotfill Poisson constant for discharges\\
$s_{wa,re}$ \dotfill See following paragraph
\end{flushright}

$s_{wa, re} \in \mathbb{W}$ is an array of unique starting weeks for each registrar in each ward. $s_{wa, re}$ is not a decision variable of the IP; rather, it is a meta-variable, enumerated by a tree with 15 unique combinations.
\end{multicols}
\subsection{Objective-Related Variables}

A is a unimodular matrix determining whether a ward is admitting on a certain day.
$$a_{wa, r} \in \{0,\ 1\} \quad\forall wa\in \mathbb{WA},\ r\in \mathbb{R}$$
O is a matrix of each ward's \emph{expected} occupancy for each day in the roster (we use expectation values to deal with the stochastic nature of ward numbers).
$$o_{wa, r} \ge 0 \quad\forall wa\in \mathbb{WA},\ r\in \mathbb{R}$$
$\dot{\text{P}}$ is the expected daily admission rate for each day of the week.
$$\dot{p_{d}} \in Z^+ \quad\forall d\in \mathbb{D}$$
$\Delta$ is the maximum pairwise difference between all wards for a given day of the roster.
$$\delta_r \ge 0 \quad\forall r \in \mathbb{R}$$

\subsection{Objective Constraints and Objective Function}

These constraints calculate the expected mean occupancy of each ward throughout the 42-day roster. $\%$ is the modulo function and $\ceil{x}$ rounds up to the nearest integer.

Control whether a ward is admitting on a given day of the 6-week cycle. The absolutely insane subscripts adjust for starting week offsets; i.e. on week one, day one, a registrar who started on week six will actually be looking at week six, day one of the roster.
\begin{align}
  a_{wa, (r+1)\%|\mathbb{R}|+1} = \sum_{re\in \{1\dots n_\text{registrars}\}}{x_{\text{A}, \ceil{\frac{r+|\mathbb{D}|\text{s}_{wa, re}-1}{|\mathbb{D}|}} \% |\mathbb{W}|+1, (r-1) \% |\mathbb{D}|+1}} \quad\forall wa \in \mathbb{WA},\ r \in \mathbb{R}
\end{align}
Calculate the expected occupancy based on the discounted occupancy from the previous day, plus the previous day's admissions if the ward was admitting. We used a mean stay of 2.1 days to model discharges as a Poisson process with a rate of 0.45.
\begin{align}
  o_{wa, r\%|\mathbb{R}|+1} = o_{wa, r} \times (1-\rho) + a_{wa, r} \times \dot{p}_{(r-1)\%|\mathbb{D}|+1} \quad\forall wa \in \mathbb{WA},\ r \in \mathbb{R}
\end{align}
For each day calculate the range of ward occupant numbers (the maximum pairwise difference).
\begin{align}
  \delta_r \ge o_{wa, r} - o_{wb, r} \quad\forall r \in \mathbb{R},\ wa \in \mathbb{WA},\ wb \in \mathbb{WA}
\end{align}

\begin{framed}
Objective function: Minimise the range of patient numbers between wards, summing over all days in the roster.
\begin{align}
  \text{minimise} \sum_{r\in \mathbb{R}}{\delta_r}
\end{align}
\end{framed}

\newpage
\appendix
\section{trump.mod}
\lstinputlisting[basicstyle={\ttfamily\small},frame=single,breaklines=true]{../trump.mod}

\end{document}
